\documentclass[11pt,a4paper,sans]{moderncv} % Font sizes: 10, 11, or 12; paper sizes: a4paper, letterpaper, a5paper, legalpaper, executivepaper or landscape; font families: sans or roman

\moderncvstyle{casual} % CV theme - options include: 'casual' (default), 'classic', 'oldstyle' and 'banking'
\moderncvcolor{blue} % CV color - options include: 'blue' (default), 'orange', 'green', 'red', 'purple', 'grey' and 'black'

\usepackage{lipsum} % Used for inserting dummy 'Lorem ipsum' text into the template

\usepackage[scale=0.75]{geometry} % Reduce document margins
\setlength{\hintscolumnwidth}{0.5cm} % Uncomment to change the width of the dates column
%\setlength{\makecvtitlenamewidth}{10cm} % For the 'classic' style, uncomment to adjust the width of the space allocated to your name

%----------------------------------------------------------------------------------------
%   NAME AND CONTACT INFORMATION SECTION
%----------------------------------------------------------------------------------------

\firstname{Panpan} % Your first name
\familyname{Xu} % Your last name

% All information in this block is optional, comment out any lines you don't need
\title{Statement of Interest}
\address{Room 2394, Academic Building, HKUST, Kowloon, Hong Kong}{}
\mobile{(852) 5170 8234}
%\phone{}
% \fax{(000) 111 1113}
\email{pxu@cse.ust.hk}
\homepage{panpanxu.org}{panpanxu.org} % The first argument is the url for the clickable link, the second argument is the url displayed in the template - this allows special characters to be displayed such as the tilde in this example
% \extrainfo{additional information}
% \photo[70pt][0.4pt]{pictures/picture} % The first bracket is the picture height, the second is the thickness of the frame around the picture (0pt for no frame)
% \quote{"A witty and playful quotation" - John Smith}

%----------------------------------------------------------------------------------------

\begin{document}



\recipient{NYU Center for Data Science}{} % Letter recipient
\date{\today} % Letter date
\opening{Dear Hiring Manager,} % Opening greeting
\closing{Sincerely yours,} % Closing phrase
\makelettertitle % Print letter title

I am writing this letter to express my interest in the research engineer position posted on the career website of your institution. I recently obtained my PhD degree in Computer Science at Hong Kong University of Science and technology. My Advisor is Prof. Huamin Qu. During the year 2014, I was also a visiting student at Georgia Institute of Technology.

During my PhD study, I developed visual analytics applications that combines data mining and interactive visualization techniques to facilitate exploration and knowledge discovery in multivariate graph data and collections of social media data. I will briefly introduce some of my research projects and industrial projects bellow. Besides that, I will also introduce my technical background and describe my past experience and my interest in open source software development. 

\section{Research Background}

\textbf{Visual analysis of social media data}

Social media data is representative of the quantity and the complexity of the data that we are facing today, and many research projects are carried out to unravel the various social process underlying the streams of posts. In this project, I developed a visual analytics system to help the domain experts to study how various topics compete to attract public attention when they spread on social media, and what roles do opinion leaders such as mass media, political figures and grassroots play in the rise and fall of various topics. The system combines text analysis, time series modeling and interactive visualization to help gain insight into the temporal dynamics of topic competition on social media (e.g.,Twitter).

\textbf{Multivariate graph data analysis}

I proposed and implemented an interactive visualization system that incorporates the co-clustering method to facilitate the identification and exploration of clusters formed by nodes with common connections in a bipartite graph. Besides that, the system also allows the users to analyze the correlation of the commonness in connection to the various features of the nodes. The visualization framework is generally applicable to any data that can be modeled as a bipartite graph. One data set I used in the case study is the roll call vote records of US senates on the passage of bills. The system helps the analysis of voting patterns of the senators and its correlation with their party affiliations and the states they represent.

\textbf{Future directions}

I believe that to harness the value of big data in the future, we need to create a closer bond between the areas of data mining, visualization, and human computer interaction. I will continue to explore the design space with different possible combinations of automatic data analysis algorithms and interactive visualizations that could effectively help the analyst to have the best of both.

\clearpage

\section{Industrial Collaboration}

I have worked closely with partners from industry in several projects, building visual analytic systems that can be combined with their data storage infrastruture and help analyse various data sets such as mobile broadband communication records and mobile user checkin data. We make design decisions by identifying their requirements, and potential areas for innovation. In some projects, I lead the design of the system architecture and coordinate the development efforts in our group.

\section{Technical Background}

In the past, I have used a lot of open source libraries to prepare and analyze data, to develop interactive visualizations, and to implement research prototypes. These include D3.js, Three.js, networkx, numpy, scikit-learn, igraph, Prefuse and so forth. Based on my own experience and observation, I believe the development of these libraries has greatly benefited and even transformed the research activities across a wide range of disciplines, and I hope to be a part of it.

Recently, I have published some open source project including a library in javascript for the creation, manipulation, and analysis of graphs as part of a research prototype, and a library that supports fast in browser density map generation for one million points in real time by harnessing the parallel processing power of GPU. They can be found on my github page (\href{http://github.com/lliquid}{http://github.com/lliquid}). I also plan to become more engaged in the open source community in the future.

From another point of view, as an information visualization researcher, I often consider one role of myself as a toolsmith who designs and engineers software applications and libraries that can lower the barrier and reduce the effort for the data enthusiasts and domain experts to create informative visualizations, and discover knowledge from the data they collected. I am also willing to contribute to the overall process of crafting the code, documenting apis, drafting tutorials, and promoting good practices. It is with great interest that I submit this letter as an application for the research engineer position.

\vspace{1.0cm}


\makeletterclosing % Print letter signature

\end{document}

